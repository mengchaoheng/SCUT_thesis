%%%%%%%%%%%%%%%%%%%此部分为附录1环境代码,是比较笨的方法来适应论文撰写规范%%%%%%%%%%%%%%%%%%%%%%%%%%%%%%%%%%%%%%
%新增附录时只需要将\setcounter{chapter}{X}以及\chapter{附\texorpdfstring{\quad}{}录 X}中相应的X更改为
%相应的数字,如果只有一个附录,则选用appendix
%%%%%%%%%%%%%%%%%%%%%%%%%%%%%%%%%%%%%%%%%%%%%%%%%%%%%%%%%%%%%%%%%%%%%%%%%%%%%%%%%%%%%%%%%%%%%%%%%%%%%%%%%%%%
\setcounter{chapter}{1} %从1开始编号
\setcounter{section}{0}
\setcounter{equation}{0}
\setcounter{table}{0}   
\setcounter{figure}{0}
\chapter{附\texorpdfstring{\quad}{}录 1} %附录1
%%%%%%%%%%%%%%%%%%%%%%%%%%%%%%%%%%%%%%%%%%%%%%%%%%%%%%%%%%%%%%%%%%%%%%%%%%%%%%%%%%%%%%%%%%%%%%%%%%%%%%%%
%%%%%%%%%%%%以下为用户代码,用于撰写您的论文%%%%%%%%%%%%%%%%%%%%%%%%%%%%%%%%%%%%%%%%%%%%%%%%%%%%%%%%%%%%%%

在论文撰写规范中,下面两段话让人费解:

\begin{enumerate}
	\item 	对需要收录于学位论文中但又不适合书写于正文中的附加数据、方案、资料、详细公式推导、计算机程序、统计表、注释等有特色的内容,可做为附录排写,序号采用“附录1”、“附录2”等。	
	\item	公式序号按章编排,如第一章第一个公式序号为“(1-1)”,附录2中的第一个公式为“(2-1)”等。
\end{enumerate}

论文撰写规范要求的附录和通常书籍上使用附录A、附录B等编号的不一样,容易和正文混淆。特殊的要求和代码的耦合,使我不得不使用比较笨的方法来设计附录部分的模板。

\section{测试一级标题 section}
\subsection{测试二级标题 subsection}
\subsubsection{测试三级标题 subsubsection}
%
测试测试测试测试测试测试测试测试测试测试测试测试测试测试测试测试测试测试测试测试测试测试测试测试测试测试测试测试测试测试测试测试测试测试测试测试测试测试测试测试测试测试测试测试测试测试测试测试测试测试测试测试测试测试测试测试测试测试测试测试测试测试测试测试测试测试测试测试测试测试测试测试测试测试测试测试测试测试测试测试测试测试测试测试测试测试测试测试测试测试测试测试测试测试测试测试测试测试测试测试测试测试测试测试测试测试测试测试测试测试测试测试测试测试测试测试测试测试测试测试测试测试测试测试测试测试测试测试测试测试测试测试测试测试测试测试测试测试测试测试测试测试测试测试测试测试测试测试测试测试测试测试测试
\begin{align}
\left\{\begin{array}{l}
\dot{v}_{1}(t)=v_{2}(t) \\
\dot{v}_{2}(t)=R^{2}\left(-\zeta_{1}\left[v_{1}(t)-v_c(t)\right]^{\alpha}-\zeta_{2}\left[\dfrac{v_{2}(t)}{R}\right]^{\beta}\right)
\end{array}\right.	
\end{align}

\begin{align}
\left\{\begin{array}{l}
\dot{v}_{1}(t)=v_{2}(t) \\
\dot{v}_{2}(t)=R^{2}\left(-\zeta_{1}\left[v_{1}(t)-v_c(t)\right]^{\alpha}-\zeta_{2}\left[\dfrac{v_{2}(t)}{R}\right]^{\beta}\right)
\end{array}\right.	
\end{align}
\begin{figure}[htbp]
	\centering	
	\includegraphics[scale=1]{Fig/DFUAV_f31.png}
	\caption{\label{fig_case1}测试测试测试}
\end{figure}
\begin{figure}[htbp]
	\centering	
	\includegraphics[scale=1]{Fig/DFUAV_f31.png}
	\caption{\label{fig_case2}测试测试测试}
\end{figure}
\begin{table}
	\caption{\label{DF_para1}测试测试测试}
	\centering{}%
	\small 
	\begin{tabular}{cccccc}
		\hline 
		参数符号 & 数值&参数符号 & 数值&参数符号 & 数值\tabularnewline
		\hline 
		$ A_x,A_y,A_z $  & $ 0.04082\,\text{m}^2 $ &$ \rho $        &$1.225\,\text{kg}/\text{m}^3$&$ I_b $           & $ 0.000029 $               \tabularnewline
		$ k_{\varpi} $   & $1.13342 \times 10^{-6}$& $ d_{\varpi} $ & $1.13342 \times 10^{-7}$ 	  &$k_{\delta} $     & $ 0.01495 $ 			      \tabularnewline
		$C_{D,x},C_{D,y}$& $ 0.43213 $             &$ C_{D,z} $     & $ 0.13421 $             	  &	$ q_a $ 	     & $ 1.49 $ 				  \tabularnewline
		$ l_{a} $        & $ -0.1121\,\text{m} $   & $ d_{ds} $     & $ 0.01495 $			  	  &$ d_{af} $        & $ 0.01495 $    			  \tabularnewline
		$ R $            & $ 0.11\,\text{m} $      &$ b $           & $ 2 $       			   	  &$ S $ 			 & $ 0.04082\,\text{m}^2 $    \tabularnewline
		$C_{l_{\alpha}}$ & $ 2.212\,/\text{rad} $  &$C_{l, \max } $ & $ 1.05 $ 				   	  &$ C_{l, \min } $  & $ -1.05 $ 				  \tabularnewline
		$ l_2 $          & $ 0.06647\,\text{m} $   &$ l_1 $         & $ 0.17078\,\text{m} $    	  &	$ m $ 		     & $ 1.53\,\text{kg} $ 		  \tabularnewline
		$ C_{d, o } $    & $ 0.9 $                 &$ C_{d, g } $   & $ 0.9 $					  &$ C_{duct} $      & $ 0.78497 $	 			  \tabularnewline
		$ I_x $          & $ 0.02548 $ 			   &$ I_y $         & $ 0.02550 $                 &$ I_z $			 & $ 0.00562 $ 				  \tabularnewline
		\hline 
	\end{tabular}	
\end{table}

\begin{table}
	\caption{\label{TDF_para2}测试测试测试}
	\centering{}%
	\small 
	%	\resizebox{\textwidth}{!}{
	\begin{tabular}{cccccc}
		\hline 
		参数符号 & 数值&参数符号 & 数值&参数符号 & 数值\tabularnewline
		\hline 
		$ I_x $ & $ 054593 $ &$ I_y $ & $ 0.017045 $& $ I_z$ & $ 0.049226 $ \tabularnewline
		$ l_{1} $ & $ 0.0808\,\text{m} $&$ l_{2} $ & $ 0.175\,\text{m} $ &$ l_3 $ & $ 0.06647\,\text{m} $ \tabularnewline 
		$ l_4 $ & $ 0.2415\,\text{m} $ &$ l_5 $ & $ 0.1085\,\text{m} $& $ m $ & $ 3.7\,\text{kg} $ \tabularnewline
		\hline 
	\end{tabular}	%}
\end{table}

\section{测试测试测试}
\subsection{测试测试测试}
%
测试测试测试测试测试测试测试测试测试测试测试测试测试测试测试测试测试测试测试测试测试测试测试测试测试测试测试测试测试测试测试测试测试测试测试测试测试测试测试测试测试测试测试测试测试测试测试测试测试测试测试测试测试测试测试测试测试测试测试测试测试测试测试测试测试测试测试测试测试测试测试测试测试测试测试测试测试测试测试测试测试测试测试测试测试测试测试测试测试测试测试测试测试测试测试测试测试测试测试测试测试测试

